\documentclass[12pt]{article}

\author{Daniel Wysocki}
\title{Hidden Markov Music}
\date{Spring 2015}


\usepackage[backend=bibtex]{biblatex}
\addbibresource{hmm.bib}

\begin{document}

\maketitle


\begin{itemize}

\item \fullcite{cope1996}

\emph{
  A classic work in AI music composition. While very different from my
  approach to the problem, it will no doubt be insightful.
}


\item \fullcite{jordanous2009}

\emph{
  A work which makes use of HMMs to implement an AI musical accompanist.
  This is in many ways similar to my project, and will provide many insights.
}


\item \fullcite{lee2006}

\emph{
  An HMM approach to identifying chords in audio. Will be useful in identifying
  chords for implementing an HMM based on higher-level musical structures.
}


\item \fullcite{li2014}

\emph{
  An HMM approach to identifying chord \emph{progressions} in audio. This is a
  layer of abstraction higher than simply identifying chords, and I suspect
  will produce much better results than simply modeling note or chord
  transition probabilities.
}


\item \fullcite{baum1966}

\emph{
  A historic paper by one of the creators of the Baum--Welsh algorithm.
  While at an unnecessarily high mathematical level for my purposes, it will
  be important to reference when making use of his algorithm.
}


\item \fullcite{stamp2012}

\emph{
  A good overview of Hidden Markov Models, explaining the underlying theory and
  the implementation and uses of the forward--backward algorithm, and the
  Baum--Walsh algorithm.
}


\item \fullcite{ibe2013}

\emph{
  A text which serves as a good general overview of Markov processes.
}




\end{itemize}


\end{document}